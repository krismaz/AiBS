\section{Methods}

We have implemented two algorithms: Exact alignment of 3 sequenses by
column based exhaustive search, and star based approximate alignment
of an arbitray number of sequenses.

Our implementation is done in Python3, using the Numpy library for
handling matrices, and the Biopython library for parsing FASTA-files.
The exact algorithm was implemented from pseudo-code from the book
\citep{Gusfield}, while the approximation algorithm was implemented
from the guidelines in the slides.

The algorithms operate on strings (instead of alternatives like arrays
or lists). They use a iterative approach as opposed to recursive
memoization.

\subsection{Running}

To run the program, \verb|python3| with the packages \verb|numpy| and
\verb|biopython| needs to be installed. On an Ubuntu-based system, this can
be achieved by running:

\verb|sudo apt-get install python3 python3-dev python3-setuptools|

\verb|sudo easy_install3 numpy biopython|

\paragraph{}
The code for exact alignment can be run with

\begin{verbatim}
python3 sp_exact_3.py <cost_file> <fasta_file> <seq1_name> <seq2_name>
 <seq3_name> [--backtrack]
\end{verbatim}

where

\begin{description}
\item{\verb|cost_file|:} A file like \verb|exc2.cost|, where the
  first line is the alphabet and the rest og the file is a cost-matrix
  (rows seperated by ; and otherwise just whitespace seperated).
\item{\verb|fasta_file|:} The name of a \verb|FASTA| file.
\item{\verb|seq[1-3]_name|:} The names of the sequenses to use from
  \verb|fasta_file|.
\item{\verb|--backtrack|:} Specifies that the algorithm should
  backtrack, i.e. actually print the alignment instead of just the
  score.
\end{description}

The code for approximate alignment can be run with

\begin{verbatim}
python3 sp_approx.py <cost_file> <fasta_file> [<seq_name>...] [--permutations]
\end{verbatim}

where

\begin{description}
\item{\verb|cost_file|:} A file like \verb|exc2.cost|, where the
  first line is the alphabet and the rest og the file is a cost-matrix
  (rows seperated by ; and otherwise just whitespace seperated).
\item{\verb|fasta_file|:} The name of a \verb|FASTA| file.
\item{\verb|seq_name|s:} The names of the sequenses to use from
  \verb|fasta_file|.
\item{\verb|--permutations|:} Specifies that the algorithm should run
  with all permutations of sequenses other than the guiding sequence,
  as required for the experiments.
\end{description}

It is important to notice, that our command-line interface does not allow specifiying the gapcost. The underlying library easily supports any positive gapcost as a parameter, but since the gapcost is fixed in all experiments, we excluded it from the external interface to avoid cluttering. 
